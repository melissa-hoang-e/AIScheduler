\documentclass[12 pt]{article}        	%sets the font to 12 pt and says this is an article (as opposed to book or other documents)
\usepackage{amsfonts, amssymb, amsmath, authblk}					% packages to get the fonts, symbols used in most math
  
%\usepackage{setspace}               		% Together with \doublespacing below allows for doublespacing of the document

\oddsidemargin=-0.5cm                 	% These three commands create the margins required for class
\setlength{\textwidth}{6.5in}         	%
\addtolength{\voffset}{-20pt}        		%
\addtolength{\headsep}{25pt}           	%



%\pagestyle{myheadings}                           	% tells LaTeX to allow you to enter information in the heading


\begin{document}

%%
%% The "title" command has an optional parameter,
%% allowing the author to define a "short title" to be used in page headers.
\title{AI-Driven Optimization for Soccer Scheduling}
\author{Melissa Hoang}
\affil{University of Calgary, Computer Science Department}
\date{\vspace{-5ex}}

%%
%% The "author" command and its associated commands are used to define
%% the authors and their affiliations.
%% Of note is the shared affiliation of the first two authors, and the
%% "authornote" and "authornotemark" commands
%% used to denote shared contribution to the research.


%%
%% By default, the full list of authors will be used in the page
%% headers. Often, this list is too long, and will overlap
%% other information printed in the page headers. This command allows
%% the author to define a more concise list
%% of authors' names for this purpose.

%%
%% The abstract is a short summary of the work to be presented in the
%% article.

%%
%% The code below is generated by the tool at http://dl.acm.org/ccs.cfm.
%% Please copy and paste the code instead of the example below.
%%



%%
%% Keywords. The author(s) should pick words that accurately describe
%% the work being presented. Separate the keywords with commas.

%% A "teaser" image appears between the author and affiliation
%% information and the body of the document, and typically spans the
%% page.

%%
%% This command processes the author and affiliation and title
%% information and builds the first part of the formatted document.
\maketitle

\section{Introduction}
n this project, we aim to address the intricate problem of scheduling soccer games and practices for various leagues and divisions within the City of Calgary. The challenge involves managing a diverse set of hard and soft constraints to ensure an optimal and feasible schedule. The hard constraints include requirements such as preventing scheduling conflicts between games and practices, adhering to the city's rule that games on Mondays must also occur at the same time on Wednesdays and Fridays (and similarly for Tuesday/Thursday games), and ensuring that only the specified game or practice slots are utilized. Soft constraints, on the other hand, involve preferences such as avoiding certain time slots, pairing specific games and practices together, and fulfilling minimum slot utilization requirements. Each game and practice has specific preferences and undesired time slots, making the problem highly combinatorial.

The objective of this project is to create a scheduling system that can take input in the form of time slot availability, game and practice requirements, and constraints, and output an optimized schedule. The system will consider penalties for violating soft constraints, such as preferences and unfulfilled pairings, while strictly adhering to the hard constraints. Our approach includes using search algorithms to generate possible schedules, evaluating them based on a penalty score, and refining the schedules to achieve the best possible balance between satisfying constraints and minimizing penalties.

We employ a hybrid approach that combines a set-based genetic algorithm with an or-tree-based search model to effectively tackle the complex scheduling problem of soccer games and practices in Calgary. The genetic algorithm is particularly well-suited for optimizing soft constraints, allowing us to explore a vast solution space while efficiently balancing multiple preferences and requirements. To address hard constraints, we incorporate an or-tree model that generates feasible solutions through systematic exploration of possible game and practice assignments. This model leverages a structured tree framework where each node represents a partial assignment, and branches explore various continuations while adhering to the defined constraints. By integrating these two methodologies, our approach can iteratively refine schedules, ensuring that games and practices are allocated to optimal time slots while minimizing conflicts and adhering to both league rules and participant preferences. This comprehensive strategy allows for a robust and adaptable scheduling solution, making it suitable for the dynamic nature of sports event management.

\section{The Model}
Model: $A = (S, T)$
The model defines the main data structure and possibilities.
Data Structure - Slots: 
\\
Given a list of slots defined as follows:
\\\begin{tabular}{|c|c|c|}
    \hline
     Games on Mon, Wed, Fri & Games on Tue, Thur   \\ \hline
     $s_1 - s_{12}$ & $s_{13} - s_{20}$  \\
     \hline
\end{tabular}
\\\begin{tabular}{|c|c|c|}
    \hline
     Practices on Mon, Wed & Practices on Tue, Thur & Practices on Fri  \\ \hline
     $s_{21} - s_{32}$ & $s_{33} - s_{44}$ & $s_{45} - s_{52}$ 
    \\ \hline
\end{tabular}

Game slots on Monday, Wednesday and Friday are represented as a tuple consisting of $s_1 - s_{12}$ such that $s_1 = MWF 9:00 - 10:00, s_2 = MWF 10:00 - 11:00$, and so on.

Game slots on Tuesday and Thursday are represented as a tuple consisting of $s_{13} - s_{20}$ such that $s_{13} = TR 8:00 - 9:30, s_{14} = TR 9:30 - 11:00$, and so on.

Practice slots on Monday and Wednesday are represented as a tuple consisting of $s_{21} - s_{32}$ such that $s_{21} = MW 9:00 - 10:00, s_{22} = MW 10:00 - 11:00$, and so on.

Practice slots on Tuesday and Thursday are represented as a tuple consisting of $s_{33} - s_{44}$ such that $s_{33} = TR 9:00 - 10:00, s_{34} = TR 10:00 - 11:00$, and so on.

Practice slots on Friday are represented as a tuple consisting of $s_{45} - s_{52}$ such that $s_{45} = F 8:00 - 10:00, s_{46} = TR 10:00 - 12:00$, and so on.

Data Structure - Games \& Practices:
\\
A tuple of games and practices, GP, is defined as follows:
$GP = (gp_1, ..., gp_n)\\
= (g_1, ..., g_m, p_{11}, ..., p_{1k}, ... p_{m1}, ..., p_{mk_m})$
\\ where $\forall j \in \mathbb{N}, 1 \le j \le m, g_j \in Games$
\\ and $\forall j' \in \mathbb{N}, 1 \le j' \le k_m, p_{jk_m} \in Practices$
\\
Facts: $F = \{ (MWF_G, TR_G, MW_P, TR_P, F_P)$ where $ MWF_G = (s_1^1, s_2^1, ...), TR_G = (s_1^2, s_2^2, ...), 
MW_P = (s_1^3, s_2^3, ...), 
TR_P = (s_1^4, s_2^4, ...), 
F_P = (S_1^5, S_2^5, ...), $
\\
$\forall j \in \mathbb{N}, 1 \le j \le |MWF_G|, s_j^1 \in Slots$ and $assign(gp_j) = s_j^1,$ \\
$\forall j \in \mathbb{N}, 1 \le j \le |TR_G|, s_j^2 \in Slots$ and $assign(gp_{|MWT_G|+ j}) = s_j^2,$ \\
$\forall j \in \mathbb{N}, 1 \le j \le |MW_P|, s_j^3 \in Slots$ and $assign(gp_{|MWT_G| + |TR_G| + j}) = s_j^3,$ \\
$\forall j \in \mathbb{N}, 1 \le j \le |TR_P|, s_j^4 \in Slots$ and $assign(gp_{|MWT_G| + |TR_G| + |MW_P| + j}) = s_j^4,$ \\
$\forall j \in \mathbb{N}, 1 \le j \le |TR_P|, s_j^5 \in Slots$ and $assign(gp_{|MWT_G| + |TR_G| + |MW_P| + |TR_P| +  j}) = s_j^5,$ \\

The set of facts it defined as the slots assigned for practices and games per week grouped together as blocks. Given our previous definition of GP, each element s in $MWF_G, TR_G, MW_P, TR_P, F_P$ corresponds to a game or practice $gp \in GP$, meaning that $assign(gp, s)$ and \\$assigned(MWF_G, TR_G, MW_P, TR_P, F_P)$.
\begin{itemize}
    \item Set of possible states: $S \subset 2^F$
    \item Transitions between states: $T = \{(S, S')|\exists A \to B \in Ext \quad with \quad A \subseteq S \quad and \quad S' = (S - A) \cup B \}$
    \item Extension Rules: $ Ext \subseteq \{ A \to B | A \subseteq F, \quad B \subseteq F, \quad Mutate(A,B) \lor SwapMutate(A,B) \lor Crossover(A,B) \lor Delete(A,B) \lor Random(A,B) \}$
    \\ The definitions of the above rules are defined below in Section 4. 
\end{itemize}

\section{Search Process}
\begin{itemize}
\item Search Process: $P = (A, Env, K)$

\item $Env: $ environment of process

\item Search Control: 
$K = S \quad x \quad  Env \to S$
\\ $K(s,e) = (s - A) \cup B \quad where \quad s \in S \quad and \quad e \in Env,$
\\ $A \to B \in Ext$
\item $f_{wert}: 2^F \quad x \quad 2^F \quad x \quad Env \to \mathbb{N}$
\\ if |s| = 0, Random(A,B)
\\ if |s| $\ge$ threshold, Delete(A,B)
\\ else
\\ 50\% of the time, Crossover (P, Q, C)
\\ \: 30\% of the time, Mutate (A, B)
\\ \: 20\% of the time, SwapMutate(A, B)
\item $f_{select}: 2^F \quad x \quad 2^F \quad x \quad Env \to 2^F \quad x \quad 2^F$
    \begin{itemize}
        \item Delete(A,B):
        \\ threshold = $
        \dfrac{\Sigma_{\forall f \in F} Penalty(f_n)}
        {|F|}$
        \\ $C = \{ f|f \in F \: and \: Penalty(f) > \: threshold \} $
        \\ A = a random 20\% of C
        \\ Threshold is the average penalty of all the assignments in A. A random 20\% of all the individuals with above-average penalty will be selected to be deleted. 
        \item Mutate(A,B)
        \\ $ A= \{P\}, B = \{P, C\}$
        \\$Ranked(L, L_{Ranked})$
        \\ Ranked is defined below
        \\ P = The first element $f$ of $L_{Ranked}$
        \\Select the individual with the lowest penalty value for mutation.
        \item SwapMutate(A,B)
        \\ $ A= \{P\}, B = \{P, C\}$
        \\$Ranked(L, L_{Ranked})$
        \\ P = The first element $f$ of $L_{Ranked}$
        \\Select the individual with the lowest penalty value for swap mutation.
        \item Crossoever(A,B)
        \\ $ A= \{P, Q\}, B = \{P, Q, K\}$
        \\$Ranked(L, L_{Ranked})$
        \\ P = The first element $f$ of $L_{Ranked}$
        \\ Q = The second element $f$ of $L_{Ranked}$
        \\Select the individuals with the lowest penalty values for crossover.
        \item Random(A,B):
        \\ Apply the or-tree search (defined in section 5) a predetermined amount of times, $N_{pop} \ge 2$ to generate $N_{pop}$ number of random unique solutions.
    \end{itemize}


\item Ranked: 
\\ $F \to (f_1, ..., f_m)$ where $\forall i \in \mathbb{N}, a \le i \le m, f_i \in F$
\\ $Ranked(L, L_{Ranked})$
\\ $L = (l_1, ..., l_k, ..., l_j, ..., l_m) \: where \: \forall i \in \mathbb{N}, 1 \le i \le m, f_i \in F$
\\ $L_{Ranked} = (..., l_j, l_k, ...)\: where \: \forall j, k \in \mathbb{N}, l_j, l_k \in L \: and \:  Pen(l_j) \le Pen(l_k)$.
\\ Given a set of facts, the facts are ranked from the lowest penalty to the highest penalty.

\item Penalty: $assigned \to \mathbb{N}$
\\Penalty(assigned(SL), pen) where SL = $(MWF_G, TR_G, MW_P, TR_P, F_P)$
    \begin{enumerate}
        \item $\forall s_i \in Slots$, if (count of $s_i$ in $MWF_G, TR_G \: < \: gamemin(s_i)$
        \\pen = pen + ($gamemin(s_i)$ -count of $s_i$ in $MWF_G, TR_G$) * $Pen_{gamemin}$
        \item $\forall s_i \in Slots$, if (count of $s_i$ in $MW_P, TR_P, F_P \: < \: practicemin(s_i)$
        \\pen = pen + ($practicemin(s_i)$ -count of $s_i$ in $MW_P, TR_P, F_P$) * $Pen_{practicemin}$
        \item $\forall s \in Slots, \forall preference((gp, s), n)$ where assigned(gp, s) = false, pen = pen + n. 
        \item $\forall pair(a,b)$ where $a,b \in Games \cup Practice$, if (assign(a) $\neq$ assign(b)) $lor$ ($\neg sametime(assign(a), assign(b))$ ), pen = pen + $pen_{notpaired}$. 
        \item $\forall g \in Games \land $ g is for tier t, $\forall g' \in Games \land g'$ is for tier t $land \: g' \neq g \land assign(g') = assign(g)$, pen = pen + $pen_{section}$. 
    \end{enumerate}
Penalty takes a possible complete assignment of all games and practices, and calculates how desirable this assignment is based on our given soft constraints. We
assume that the model allows only positive numbers as penalties. Therefore, the
lower the penalty number is, the more ”fit” the assignment is.
\begin{enumerate}
    \item If the number of games assigned to a slot is lower than the slot’s gamemin,
then we add the number it is lower by, times the predetermined penalty score
to the existing penalty number.
    \item If the number of practices assigned to a slot is lower than the slot’s practicemin,
then we add the number it is lower by, times the predetermined penalty score
to the existing penalty number.
    \item Given a list of preferences of games/practices for slots, for every preference of
an assignment that is not met, we add the predetermined penalty score of that
preference to the existing penalty number.
    \item Given a list of pairs consisting of games/practices that would like to be scheduled
to the same slot, for every pair that is not scheduled to the same slot, we
add the predetermined penalty score to the existing penalty number.
    \item For all games of a tier t, if there exists another game $g'$ that is scheduled to the same slot, but is in a different division and the same tier, for every $g'$ we add the predetermined penalty score.
\end{enumerate}
\item assigned: $f \to \{true, false\}, where \: f \in F$
\\assign(A) = true if and only if 
\\ A = $(MWF_G, TR_G, MW_P, TR_P, F_P)$ = $((s_{l_1}, ..., s_{l_i}), (s_{l_{i+1}}, ..., s_{l_k}), (s_{l_{k+1}}, ..., s_{l_l}), (s_{l_{l+1}}, ..., s_{l_m}))$
\\ such that $assign(gp_1) = s_{l_1}, ..., assign(gp_m) = s_{l_m}$.
\\GP = $(gp_1, ..., gp_m)$ is defined in section 2.
\\Given an A consisting of tuples of slots, assigned(A) is true if each slot is assigned to its corresponding game or practice as defined in section 2. 
\end{itemize}

\section{Rule Definitions}
\begin{itemize}
    \item Crossover: Crossover(A,B) where
    \\$A= \{P, Q\}, B = \{P, Q, K\}$, where for some $i, j, m \in \mathbb{N}, 1\le i, j, m \le |P|, |Q|, P = (s_1, ..., s_i, ..., s_j)$ and
    \\ $Q = (s_1, ..., s_i, ..., s_j, ...)$ and
    \\ $gp_m \in Q$ such that $gp_m$ is assigned to $s_i$ and $gp_m$ is assigned to $s_j$. 
    \\ $P \neq Q$
    \\ If $gp_m$ is chosen, then for Q, $assign(gp_m) = s_i$, and for P, $assign(gp_m) = s_j, K = \{(s_1 ... s_i ... s_j), (s_1 ... s_i ... s_j)\}$
    \\Pass K to the or-tree to check if it meets hard constraints, if not let or-tree modifies K to the next
viable solution that’s not equal to P or Q.
    \\Crossover takes two parent individuals, and randomly selects which game/practice to crossover, and
places the game/practice from one parent in the slot that the other parent has the game/practice
in, and the other parent does the same, placing the game/practice in the corresponding slot to that
that contained the practice/game in the other parent.The practice/game chosen will be done so
randomly.
    \item Swap Mutation: SwapMutate(A,B) where
    \\ $A = \{P\}, B = \{P, C\}
    \\ P \in F, P = (s_1 ... s_i ... s_j ...)$ such that
    \\ $assign(gp_i) = s_j$ and $assign(gp_j) = s_i$ and Constr(C) = true.
    \\ Pass C to the or-tree to check if it meets hard constraints, if not let or-tree modifies C to the next
viable solution that’s not equal to P.
    \\ SwapMutation takes a parent individual from A and swap the slot between 2 slots
or 2 practices. The result of the swap is appended to the list.
\item Mutation: Mutate(A, B) where
\\ $A = \{P\}, B = \{P,C\}$
\\ $P \in F, P = (s_1 ... s_i ... s_j)$ such that
\\ $assign(gp_i) = s_i $ and $assign(gp_j) = s_j$ for some $i, j \in \mathbb{N}, 1 \le i \le |P|, 1 \le j \le |P|, j \neq i$ and
\\ $((gp_i \in Games) \land (gp|j \in Games)) \lor ((gp_i \in Practices) \land (gp_j \in Practices))$
\\ $C = s_1 ... s_j ... s_j ...$ such that $assign(gp_i) = s_j, Constr(C) = true$.
\\Pass C to the or-tree to check if it meets hard constraints, if not let or-tree modifies C to the next
viable solution that’s not equal to P.
\\Mutation takes a parent individual from A and reassign the slot of a game or
practice of the parent by selecting another slot assigned to some other game or
practice. The result of the reassignment is appended to the list.
\item Deletion: Delete(A,B) where
\\ $A \subset F, B = \{\}$
\\Deletion takes a set of facts and remove them.
\item Random: Random(A, B) where
\\ Apply the Or-tree search a predetermined amount of times, $N_{pop} \ge 2$, to generate $N_{pop}$ number of
random unique solutions.
\\ B = A passed down to the or-tree $N_{pop}$ times.
\end{itemize}

\section{Random: Or-tree Model}
\subsection{Or-tree-based Search Model}
For our Random rule, we would like to use an or-tree model to populate individuals that meet the
hard constraints while disregarding the soft constraints.
\begin{itemize}
    \item Model $A_{\lor} = (S_{\lor}, T_{\lor}$
    \item Prob: set of problem descirptions
    \\ $pr \in Prob $ where $pr = WMF_G, TR_G, MW_P, TR_P, F_P$ where $ MWF_G = (s_1^1, s_2^1, ...), TR_G = (s_1^2, s_2^2, ...), 
MW_P = (s_1^3, s_2^3, ...), 
TR_P = (s_1^4, s_2^4, ...), 
F_P = (S_1^5, S_2^5, ...), $
\\
$\forall j \in \mathbb{N}, 1 \le j \le |MWF_G|, s_j^1 \in Slots \cup \$$  \\
$\forall j \in \mathbb{N}, 1 \le j \le |TR_G|, s_j^2 \in Slots \cup \$$  \\
$\forall j \in \mathbb{N}, 1 \le j \le |MW_P|, s_j^3 \in Slots \cup \$$  \\
$\forall j \in \mathbb{N}, 1 \le j \le |TR_P|, s_j^4 \in Slots \cup \$$ \\
$\forall j \in \mathbb{N}, 1 \le j \le |TR_P|, s_j^5 \in Slots \cup \$$ . 
\\ These are the same definitions for the Facts as in section 2, except that the s1, s2, s3, s4, and s5 can now be \$. \$ indicates that the game / practice is not assigned to any slot yet.
\item $Altern \subseteq Prob^+:$: alternatives relation.
\\ $Altern(pr, pr_1, ..., pr_n)$ where
\\ $\forall s \in pr, s_i is the first such that \$ \in s_i$, and
\\ $\forall s_j \in Slots, \exists pr_j$ that is created by duplicating pr and then replace the first \$ in pr by $s_j$. s
\item $S_{\lor} \subseteq Otree$: set of possible states, is subset tree structures where 
\\ $(pr, sol) \in Otree$ for $pr\in Prob$, $sol \in \{ues, ?, no\}, b_i \in Otree$
\item $T_{\lor} \subseteq S_{\lor} \: x \: S_{\lor}$: transitions between states such that 
\\ $T_{\lor} = \P(s_1, s_2|s_1, s_2 \in S_{\lor}$ and $Erw(s_1, s_2)$
\item $Erw_{\lor}$: A relation on Otree
\begin{itemize}
    \item $Erw_{\lor}((pr, ?), (pr, yes)$ if pr is solved
    \item $Erw_{\lor}((pr, ?), (pr, no)$ if pr is unsolvable
    \item $Erw_{\lor}((pr, ?), (pr, (pr_1, ?), ..., (pr_n, ?)))$ if $Altern(pr, pr_1, ..., pr_n)$ holds
    \item $Erw_{\lor}((pr, ?, b_1, ..., b_n), (pr, ?, b_{1}^{'}, ..., b_{n}^{'}))$ if for an i, $Erq_{\lor}(b_i, b_{i}^{'})$ and $b_j = b_{j}^{'}$ for $i \neq j$
\end{itemize}
\end{itemize}
\subsubsection{Or-tree-based Search Process}
\begin{itemize}
    \item Search Process: $P_{\lor} = (A_{\lor}, Env, K_{\lor}$
    \\Env: environment of process
    \item Search Control: $K: X \: x \: Env \to S$
    \begin{itemize}
        \item $K(s,e) = s^{'} \: where \: (s, s^{'}) \in T, e \in Env$
        \item Let $(pr_1, ?), ..., (pr_0, ?)$ be the open leafs in the current state, and $X = (MWF_G, ..., F_P)$ with the same definitions in section 2 such that X is over some discrete value domains $D = \{D_i | i \in \mathbb{N}, 1 \leq i \leq |X| \}$
        $\forall i \in \mathbb{N}, 1 \leq i \leq |X|, D_i = Slots$.
        \\ $C = \{C_1, ..., C_m\} $ is a set of constraints. Each constraint $C_i$ is a relation of a subset of the vairables, i.e.
        \\ $C_i = R_i(s_{i,1}, ..., s_{i,k})$ where the relation $R_i$ descirbes every value-tupe in $D_{i, x} \: x \: ... \: x \: D_{i,k} $
        \\ Let $const(X_{m_i}) = |\{C_i|C_i \in C, x_1^1, ..., x_n^5 \: fulfills \: C_i\}|$ 
        \item $C_{solved(pr)} = |\{C_i|C_i \in C, x_1^1, ..., x_n^5 \: fulfills \: C_i \}|$
        \item Define constraints $C = \{C_1, ..., C_{12}\}$
        \begin{enumerate}
            \item $C_1: \forall x \in MWF_G \cup TR_G, \ \text{count of}(x) = s \leq \text{gamemax}(s), \ \text{for} \ s \in \text{Slots}.$
            \item $C_2: \forall x \in MW_P \cup TR_P \cup F_P, \ \text{count of}(x) = s \leq \text{practicemax}(s), \ \text{for} \ s \in \text{Slots}.$
            \item $C_3: \forall x = g_i \ \text{for Division} \ i \in MWF_G \cup TR_G, \ \not\exists x' = p_{ik_i} \in MW_P \cup TR_P \cup F_P, \ \text{such that} \ \text{assign}(x) = \text{assign}(x').$
            \item $C_4: \forall \ \text{not compatible}(a, b) \ \text{where} \ x = a, \ x' = b, \ a, b \in Games \cup Practices, \ \text{assign}(x) \neq \text{assign}(x').$
            \item $C_5: \forall \ \text{unwanted}(a, s) \ \text{where} \ a \in Games \cup Practices \ \text{and} \ s \in Slots, \ \not\exists x = a \ \text{such that} \ \text{assign}(a) = s.$
            \item $C_6: \forall x \in Games \cup Practices, \ \text{if Div of } x \geq 9, \ \text{assign}(x) = s \ \text{such that} \ s \ \text{is in the evening.}$
            \item $C_7: \forall x \in Games \ \text{where tier of} \ x \ \text{is U15/16/17/19}, \ \not\exists x' \in Games, \ x' \neq x, \ \text{and tier of} \ x' \ \text{is also U15/16/17/19}.$
            \item $C_8: \forall x \in X, \ \not\exists \ \text{assign}(x) = s \ \text{such that} \ s = TR \ 11:00 - 12:30.$
            \item $C_9: \forall x \in X, \ x \in Games, \ \text{if} \ (x = CMSA \ U12T1S \lor x = CMSA \ U13T1S) \rightarrow \text{assign}(x) = TR \ 18:00 - 19:00.$
            \item $C_{10}: \forall x \in X, \ \text{if} \ x \ \text{is for} \ CMSA \ U12T1 \land x \neq CMSA \ U12T1S \rightarrow \text{assign}(x) \neq TR \ 18:00 - 19:00.$
            \item $C_{11}: \forall x \in X, \ \text{if} \ x \ \text{is for} \ CMSA \ U13T1 \land x \neq CMSA \ U13T1S \rightarrow \text{assign}(x) \neq TR \ 18:00 - 19:00.$
            \item $C_{12}: \forall x = g_i \in Games, \ \forall x' = p_{ik}, $\\ $\text{assign}(x) = s_i, \ \text{assign}(x') = s_j \ \text{such that} \ \text{same time}(s_i, s_j) = \text{false}.$
        \end{enumerate}
        \item If one of the $pr_j$ is solved, perform the transition that changes its sol-entry. If there are several, select one of them randomly.
        \item Else if one of the $pr_j$ is unsolvable, perform the transition that changes its sol-entry. If there are several, again select one of them randomly.
        \item Else, select the leaf($pr_j$ , ?) such that
        \begin{enumerate}
            \item $C_{\text{solved}}(pr_j) = \max_{pr_l}(\{C_{\text{solved}}(pr_l)\})$
            \item \text{If there are several, select the deepest leaf in the tree with this property.}
            \item \text{If there are still several, select the one most left in the tree.}
        \end{enumerate} 
    \end{itemize}
\end{itemize}
\subsubsection{Or-tree-based Search Instance}
\begin{itemize}
    \item Search Instance: $Ins_{\lor} = (s_0, G_{\lor})$
    \item $s_0 = \{partassign\}$
    \\ if the given problem to solve is pr, then we have:
    \begin{itemize}
        \item $s = (pr^{'}, yes)$ or
        \item $s = (pr^{'}, ?, b_1, ..., b_n), G_{\lor}(b_i) = yes$ for an i, or
        \item All leafs of s have either the sol-entry no, or cannot be processed using $Altern$
    \end{itemize}
\end{itemize}
\section{Search Instance}
The search instance of our model defines the initial state $s_0$ as an empty set, which will be passed
into our model. The goal condition decides that the search terminates when we find a solution that
has a desired penalty score, $score_{desired}$, when we have applied the extension rule $pop_{max}$ times, or when there’s no more rules to apply in our model. The solution with the desired score is returned.
For the case that a desired score is not reached, return the solution with the lowest penalty.
\begin{itemize}
    \item \textbf{Search Instance:} $\text{Ins} = (s_0, G)$
    \item \textbf{Define } $s_0: \ s_0 = \{\}$
    \item \textbf{Define } $s_{\text{goal}}: \ s_{\text{goal}} \in 2^F \ \text{such that}  \exists f_i \in s_{\text{goal}} \ \text{where} \ f_i \ \text{such that}$
    \[
    \text{Penalty}(\text{assigned}(GP, f_i)) \leq \text{Score}_{\text{desired}}
    \]
    \\
    $\text{Score}_{\text{desired}} $ is a predefined constant that will be determined during the implementation phase. \\ This goal condition is in place to improve efficiency in the case that a desired solution is
    identified early.
    
    \item \textbf{Goal condition:} $G: S \to \{\text{yes}, \text{no}\}$
    \[
    G(s_i) = \text{yes} \ \text{if and only if} \ s_{\text{goal}} \subseteq s_i, \ \text{or}
    \]
    \[
    \text{there is no extension rule applicable in} \ s_i, \ \text{or}
    \]
    \[
    \text{when we reach the final iteration of extension.}
    \]
    \item The final iteration of extension limits the number of times we populate new generations. 
    \\ This is predefined with a constant value  $pop_{\text{max}} \in \mathbb{N}$.
    
    \item When we reach the final iteration, set the  $s_{\text{goal}}$ to contain the individual with the lowest penalty; if there are multiple, the left-most one. \\ Let $s_{\text{goal}} = \{f\} \in s_i$ \text{ where } 
    $\not\exists g \in s_i \ \text{such that Penalty}(g) < \text{Penalty}(f)$  
    
\end{itemize}

\end{document}